\section{Related Work}\label{sec:related}

\textbf{Cloud Resource Provisioning}. Many existing works have proposed to use predicative auto-scaling for dynamic resource provisioning in cloud computing. In \cite{niu2011demand, niu2012quality}, a statistical model is used to predict the demand of videos, based on which the video service provider can dynamically book bandwidth resources to match the fluctuated demand. In \cite{wu2012scaling, wu2015scaling}, an epidemic model is built to forecast the viewing requests in a social media application. Similar predictive auto-scaling models have also been built for resource allocation \cite{tang2007scalable, gong2010press, wang2011consolidating} and power consumption \cite{kusic2009power, lin2013dynamic} in cloud systems. In this paper, we determine the backend capacity based on predicted request demand, but our work distinguishes from previous works in two aspects. First, there is a mismatch between timescales of request demand fluctuations and backend capacity adjustment, so that fine-grained auto-scaling is infeasible.  Second, unlike the rigid demand in existing models, we can exploit the delay tolerance of requests to change the demand profile and help the application developer cut cost. This is similar to \cite{ha2012tube}, in which pricing is used to incentivize users to shift their demand. But rather than relying on users' subjective decisions, our request scheduling mechanism is directly controlled by Clockwork.       


\textbf{NUM-based Resource Allocation}. The pioneering work of Kelly et al. \cite{kelly1998rate} first introduced the novel idea of Network Utility Maximization (NUM) based resource allocation. A nice survey of the research on network utility maximization problem is given by \cite{yi2008stochastic}. Mo and Walrand \cite{mo2000fair} first introduced the $\alpha$-fair utility function to be used as the objective function. NUM-based network resource allocation has been widely applied to rate allocation in Internet congestion control protocol \cite{chiang2006stochastic}, resource allocation in cellular networks \cite{stolyar2005asymptotic}, and congestion control in wireless ad hoc networks \cite{neely2008fairness}. We build a similar model but relate it to request scheduling in mobile applications. In particular, we leverage the model to allocate rates to users according to the quantity and delay tolerance of their requests.     
